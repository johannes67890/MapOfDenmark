\newpage
\section{Appendix}
\subsection*{A Formal Requirements}
\subsubsection*{A.1}
\begin{enumerate}
    \item Allow the user to specify a data file for the application. The system should support loading a zipped .OSM file, but you may support other formats as well.
\item Include a default binary file embedded as a resource of the program, which is loaded if the user does not choose another input file at start-up.
\item Draw all roads in the loaded map dataset, using different colors for the different types of roads in the dataset.
\item Show any rectangle of the map, as indicated by the user.
\item Show the current zoom level, e.g., in the form of a graphical scale bar.
\item Allow the user to change the visual appearance of the map, e.g., toggle color blind mode, or customize which elements to show.
\item Adjust the layout when the size of the window changes.
\item Ensure a clean GUI which is reasonably easy to navigate.
\item Allow the user to search for addresses typed as a single string, and show the results on the map. The program must handle ambiguous user input appropriately, e.g., by displaying a list of possible matches.
\item Make the computation of a shortest route on the map between two points for specified by the user (e.g., by typing addresses or clicking on the map).
\item Allow the user to choose between at least 2 types of routes: for example, for biking routes and cars routes. Car routes should take into account speed limits/expected average speeds. A route for biking/walking should be the shortest and cannot use highways.
\item Allow the user to add something to the map, e.g., points of interest.
\item Be fast enough for convenient use, even when working with a complete data set for Denmark. The load time for .OSM can be hard to improve, but loading a binary file should be fast. After the start-up the user interface should respond reasonably smooth.
\end{enumerate}
\subsection*{B Group of Conduct}
\subsubsection*{B.1}
\textbf{The working policy for the group.}
\begin{enumerate}
\item Any breach of the following rules will result in cake-penalty.
\item Academic quarter.
\item Don’t be late.
\end{enumerate}
\textbf{The frequency with which you will meet each other.}
\newline
2 times a week at a minimum
\newline
Usually Monday after lectures and the other day will depend on when the TA decides to have his meeting. So at the moment: Monday, Tuesday.
\newline
Bufferday if needed: Friday before lectures.
\newline
\textbf{Where you will work (normally)}
\newline
2 times a week at the school or another physical place. 1-2 times a week from home.
\newline
\textbf{How well as a group do you plan to do in this course.}
\newline
We don’t have any goal we strive for academically. But we do have a goal on having a great time with the project which will probably result in average performance. We will try our best, but not stress over striving for an exceptional grade.
\newline
\textbf{Progress plan for the first two weeks.}
\newline
 Github workflow, Discuss the idea of group rotations with last member, Contact David.
\newpage
\subsection*{C Log}
\subsubsection*{C.1}
\hypertarget{monday-04-03-2024}{%
\subsubsection*{Monday 04-03-2024}\label{monday-04-03-2024}}

\emph{Meeting time: 12:00-16:00}\\
We started looking into the project description, and have discussed the
different requirements for the project. We have started discussing
different ways to render the map in chunks. We have discussed how
relations outside our view-box could affect what chunks are rendered,
and how we should render them. We decided that this discussion would be
better suited for later.

We used a lot of time trying to get a simple version of the program
running on all of our machines. We had some issues with the people using
IntelliJ not being able to run the program, while the Visual Studio Code
users had none of those issues. People were generally fine with
switching to VSCode to alleviate any issues, although we did try to fix
the issues with IntelliJ.

\hypertarget{tuesday-05-03-2024}{%
\subsubsection*{Tuesday 05-03-2024}\label{tuesday-05-03-2024}}

\emph{Meeting time: 10:00-16:00}\\
The first meeting was made with our guidance counsellor, David. As much
has yet to be implemented in our system, we mainly spoke about what was
about to come in the following weeks, as well as what to expect in the
exam.

After the meeting, we mainly focused on the technical aspects of our
work environment. Ensuring that our base project works on all computers,
and ensuring that Visual Studio Code handles javaFX correctly.

Otherwise, implementations have not been our highest priority, however,
the parse related features have been further developed, and some
previous features were moved from intelliJ projects into various
branches in our git.

\hypertarget{thursday-07-03-2024}{%
\subsubsection*{Thursday 07-03-2024}\label{thursday-07-03-2024}}

\emph{Meeting time: 10:00-14:30}\\
Today we worked from home, at 10:00 we had a meeting were we discussed
the schedule of the day. After that we split up into 3 groups and worked
on these issues: Regex, GUI and XML Parsing. We called it at 14:30 ish.

\hypertarget{monday-11-03-2024}{%
\subsubsection*{Monday 11-03-2024}\label{monday-11-03-2024}}

\emph{Meeting time: 12:00-14:00}\\
Under this meeting we concluded on using milestones within git forward
on. By using milestones we can set some clear boundaries for what we
have to achieve every week and by knowing that having the ability to
coordinate or estimate when we will take up issue \#x.

In this week we will finish up making the first revision of the GUI for
the map, Regex- and XML-parser.

\hypertarget{tuesday-12-03-2024}{%
\subsubsection*{Tuesday 12-03-2024}\label{tuesday-12-03-2024}}

Minor changes to the parser and refactoring.

\emph{Meeting time: 10:00-14:00}\\
Due to disease, two of our members were unable to attend todays meeting.
We had a meeting with our TA, where we talked about GANT diagrams, and
spoke about some of the theory from a prior course we could make use of.
We worked on implementing the Levenshtein distance algorithm to give us
a sort of ``autocorrection'' in case the user made a spelling mistake
when searching for an address. We implemented matching input with city
and street names. We also refactored the parser.

\hypertarget{thursday-14-03-2024}{%
\subsubsection*{Thursday 14-03-2024}\label{thursday-14-03-2024}}

\emph{Meeting time: 12:00-16:00}\\
As we had a lecture, most of the group met up afterwards, although some
members were prevented from being physically being at ITU. We split up
in the same three groups as last Thursday, and worked some of the same
things. We fixed bug so address parsing is possible. We also added types
for OSM data.

\hypertarget{monday-18-03-2024}{%
\subsubsection*{Monday 18-03-2024}\label{monday-18-03-2024}}

\emph{Meeting time: 12:00- 16:00 }\\
Our meeting began with us having an internal meeting about what we
accomplished and to what extend we did it last week. Then we talked
about whats next, which issues would make the most sense to solve at the
moment and after that we delegated the work out between us.

\hypertarget{tuesday-19-03-2024}{%
\subsubsection*{Tuesday 19-03-2024}\label{tuesday-19-03-2024}}

\emph{Meeting time: 10:00- 16:00 }\\
We started the day off with a meeting with our TA. We got some feedback
on the project structure and guidance on what our different discussed
ideas would benefit the project the most to prioritize. After that we
had a clear goal in mind for the week. To display Bornholm.

\hypertarget{thursday-21-03-2024}{%
\subsubsection*{Thursday 21-03-2024}\label{thursday-21-03-2024}}

\emph{Meeting time: 10:00- 16:00 }\\
Make the \texttt{XMLWriter} to initilize chunk files.\\
Other changes to how the map is displayed and more.

\hypertarget{tuesday-0204-2024}{%
\subsubsection*{Tuesday 02/04-2024}\label{tuesday-0204-2024}}

\emph{Meeting time: 10:00- 16:00 }\\
The day started off with us having the first progress presentation. The
other groups presentations solved some roadblocks or inspired us to do
things different. Things that we began working on after this meeting
was, GUI-rebuild, K-D Trees, Weighting of different elements which will
be used to decide if the element should be displayed at a given
zoom-level.

\hypertarget{thursday-0404-2024}{%
\subsubsection*{Thursday 04/04-2024}\label{thursday-0404-2024}}

\emph{Meeting time: 10:00- 16:00 }\\
Switched to FXML for the UI. Added a few new types to parsed elements.

\hypertarget{monday-0804-2024}{%
\subsubsection*{Monday 08/04-2024}\label{monday-0804-2024}}

\emph{Meeting time: 12:00- 16:00 }\\
KD-tree group from home, agreed on what to prioritize this week.
Kd-trees, relations, filestreaming, adress-searching.

\hypertarget{tuesday-0904-2024}{%
\subsubsection*{Tuesday 09/04-2024}\label{tuesday-0904-2024}}

\emph{Meeting time: 10:00- 16:00 }\\
Missed meeting, with TA David. But stilled worked together on tuesday.

\hypertarget{thursday-1104-2024}{%
\subsubsection*{Thursday 11/04-2024}\label{thursday-1104-2024}}

\emph{Meeting time: 10:00- 16:00 }\\
We worked from home, had a meeting at 10. ish. and then split up into
our groups. Kd-Trees, Relations, Map-painting, File-streamer.

\hypertarget{monday-1504-2024}{%
\subsubsection*{Monday 15/04-2024}\label{monday-1504-2024}}

\emph{Meeting time: 12:00- 16:00 }\\
Grouped up and agreed on to do the following this week. - Implementation
of custom collections library, that takes primitive datatypes directly.
So no reference of object, instead just object. - Adress-searching and
UI - Pathfinding-implementation - Finishing up KD-trees. Closest node
and comment/documentation

\hypertarget{tuesday-1604-2024}{%
\subsubsection*{Tuesday 16/04-2024}\label{tuesday-1604-2024}}

\emph{Meeting time: 10:00- 16:00 }\\
Fix major bugs in the Mercator Projection code. The core problem was
that we didn't negate the latitude in the parser, so originally we did
it in the Mercator Projector. This was a mistake because it messed up
calculations with everything else.\\
This also came with a small refactor of the \texttt{DrawMap} class.

\hypertarget{wednesday-1704-2024}{%
\subsubsection*{Wednesday 17/04-2024}\label{wednesday-1704-2024}}

\emph{Meeting time: 10:00- XX:XX }\\
On Wednesday we experimented with making \texttt{TagWay}'s as a
linked-list, that could benefit us in the long run.\\
This will mean hopefully better chunking capabilities and more. \#
Wednesday 17/04-2024

\hypertarget{thursday-1804-2024}{%
\subsubsection*{Thursday 18/04-2024}\label{thursday-1804-2024}}

\emph{Meeting time: 10:00- XX:XX }\\
Used most of the day to refactor code in Mecator Projection and other
parts of the code. Also a lot of focus for memory optimization.

\hypertarget{monday-2204-2024}{%
\subsubsection*{Monday 22/04-2024}\label{monday-2204-2024}}

\emph{Meeting time: 10:00- 16:00 }\\
We merged a lot of code into main from the previous week.~ Work-on
KDTree and Chunks was also in focus, where we fixed alot of problems
with writeing to chunks and improved the KDTree to be used later.

\hypertarget{tuesday-2304-2024}{%
\subsubsection*{Tuesday 23/04-2024}\label{tuesday-2304-2024}}

\emph{Meeting time: 10:00- 16:00 }\\
Start to work on 3D KDTrees, where the their dimension is for a
hierarchy system for what to draw and when.\\
Start to workon how to read from chunks.\\
Achived functional address searching.

\hypertarget{thursday-2504-2024}{%
\subsubsection*{Thursday 25/04-2024}\label{thursday-2504-2024}}

\emph{Meeting time: 10:00- 16:00 }\\
Made the complete switch to linked-list based \texttt{TagWay}'s.\\
Memory improvements and a new feature for a zoom-bar has been made.

\hypertarget{monday-2904-2024}{%
\subsubsection*{Monday 29/04-2024}\label{monday-2904-2024}}

\emph{Meeting time: 10:00- 16:00 }\\
Working to improve the searching for an address and more!\\
We also try to use a new tree data structure for searching, the
structure is \texttt{Trie}.

\hypertarget{tuesday-3004-2024}{%
\subsubsection*{Tuesday 30/04-2024}\label{tuesday-3004-2024}}

\emph{Meeting time: 10:00- 19:00 }\\
Trie is now fully functional, and allows us to search for a given
adddress, where the program will autocomplete throughout. Additional
development has been made on pathfind, which now has had a working
instance.

\hypertarget{wednesday-0105-2024}{%
\subsubsection*{Wednesday 01/05-2024}\label{wednesday-0105-2024}}

\emph{Meeting time: 10:00- 22:00 }\\
New color palette, inspired by OpenStreetMap's color theme, and added
more ui for searching and pathfinding.

\hypertarget{thursday-0205-2024}{%
\subsubsection*{Thursday 02/05-2024}\label{thursday-0205-2024}}

\emph{Meeting time: 10:00- 18:00 }\\
Functional incorporation with a*.

\hypertarget{friday-0305-2024}{%
\subsubsection*{Friday 03/05-2024}\label{friday-0305-2024}}

\emph{Meeting time: 10:00- 23:00 }\\
Tests added, refactoring directory.

\hypertarget{saturday-0405-2024}{%
\subsubsection*{Saturday 04/05-2024}\label{saturday-0405-2024}}

\emph{Meeting time: 10:00- 19:00 }\\
Most branches merged into main.

\hypertarget{sunday-0505-2024}{%
\subsubsection*{Sunday 05/05-2024}\label{sunday-0505-2024}}

\emph{Meeting time: 10:00- 20:00 }\\
Documentation, tests, css style.

\hypertarget{monday-0605-2024}{%
\subsubsection*{Monday 06/05-2024}\label{monday-0605-2024}}

\emph{Meeting time: 12:00- 01:30 }\\
Final changes, testing, functional zipping.
\newpage
\subsection*{D Git Log}
\subsubsection*{D.1}

\hypertarget{officiel-release-version-1.0}{%
\subsubsection*{Officiel Release version
1.0}\label{officiel-release-version-1.0}}

\hypertarget{whats-changed}{%
\subsubsection*{What's Changed}\label{whats-changed}}

\begin{itemize}
\tightlist
\item
  init project by @johannes67890 in\\
  https://github.com/johannes67890/MinionMap/pull/1
\item
  Java fx init by @johannes67890 in\\
  https://github.com/johannes67890/MinionMap/pull/15
\item
  Initial OSMParser by @johannes67890 in\\
  https://github.com/johannes67890/MinionMap/pull/21
\item
  Address search by @johannes67890 in\\
  https://github.com/johannes67890/MinionMap/pull/23
\item
  GUI by @johannes67890 in\\
  https://github.com/johannes67890/MinionMap/pull/25
\item
  Address search integration by @Spurberino in\\
  https://github.com/johannes67890/MinionMap/pull/28
\item
  Map drawing by @AndreasLN in\\
  https://github.com/johannes67890/MinionMap/pull/31
\item
  Add MecatorProjection class and update TagNode constructor by\\
  @johannes67890 in https://github.com/johannes67890/MinionMap/pull/32
\item
  Zip file and pathfinder implementation by @MessiGames30 in\\
  https://github.com/johannes67890/MinionMap/pull/30
\item
  Merge from main by @AndreasLN in\\
  https://github.com/johannes67890/MinionMap/pull/33
\item
  Merge with FXML rebuild by @Hopsasasa in\\
  https://github.com/johannes67890/MinionMap/pull/37
\item
  OSMParser Version 2 by @johannes67890 in\\
  https://github.com/johannes67890/MinionMap/pull/38
\item
  Osm parser v2 by @johannes67890 in\\
  https://github.com/johannes67890/MinionMap/pull/39
\item
  Filestreaming by @johannes67890 in\\
  https://github.com/johannes67890/MinionMap/pull/40
\item
  Import alg4 lib by @AndreasLN in\\
  https://github.com/johannes67890/MinionMap/pull/42
\item
  Map coloring relations by @AndreasLN in\\
  https://github.com/johannes67890/MinionMap/pull/43
\item
  K d tree implementation by @MessiGames30 in\\
  https://github.com/johannes67890/MinionMap/pull/44
\item
  Map coloring relations by @johannes67890 in\\
  https://github.com/johannes67890/MinionMap/pull/45
\item
  Zoom scale made, comments made, want to merge into main by\\
  @MessiGames30 in https://github.com/johannes67890/MinionMap/pull/47
\item
  Macatorprojection fix by @johannes67890 in\\
  https://github.com/johannes67890/MinionMap/pull/49
\item
  KDTree implementation with Mecator by @Hopsasasa in\\
  https://github.com/johannes67890/MinionMap/pull/50
\item
  Memory opt by @johannes67890 in\\
  https://github.com/johannes67890/MinionMap/pull/51
\item
  Merge pull request \#51 from johannes67890/memoryOpt by @AndreasLN in\\
  https://github.com/johannes67890/MinionMap/pull/52
\item
  Kd tree optimization into main by @AndreasLN in\\
  https://github.com/johannes67890/MinionMap/pull/53
\item
  main into search by @AndreasLN in\\
  https://github.com/johannes67890/MinionMap/pull/55
\item
  Kd tree memory opt by @johannes67890 in\\
  https://github.com/johannes67890/MinionMap/pull/57
\item
  Zoombar scaling by @MessiGames30 in\\
  https://github.com/johannes67890/MinionMap/pull/56
\item
  relations fix into main by @AndreasLN in\\
  https://github.com/johannes67890/MinionMap/pull/58
\item
  Color optimization into main by @AndreasLN in\\
  https://github.com/johannes67890/MinionMap/pull/59
\item
  Updated when to load smaller streets by @Spurberino in\\
  https://github.com/johannes67890/MinionMap/pull/60
\item
  Zoombar fixed by @MessiGames30 in\\
  https://github.com/johannes67890/MinionMap/pull/61
\item
  Zoominonrelation into main by @AndreasLN in\\
  https://github.com/johannes67890/MinionMap/pull/62
\item
  Testing into main by @AndreasLN in\\
  https://github.com/johannes67890/MinionMap/pull/63
\item
  UI by @johannes67890 in\\
  https://github.com/johannes67890/MinionMap/pull/64
\item
  Docs by @johannes67890 in\\
  https://github.com/johannes67890/MinionMap/pull/65
\item
  Swap button fix by @johannes67890 in\\
  https://github.com/johannes67890/MinionMap/pull/66
\item
  Bin last chance by @johannes67890 in\\
  https://github.com/johannes67890/MinionMap/pull/67
\item
  Zip file fix by @johannes67890 in\\
  https://github.com/johannes67890/MinionMap/pull/68
\item
  Tests by @johannes67890 in\\
  https://github.com/johannes67890/MinionMap/pull/69
\end{itemize}

\hypertarget{new-contributors}{%
\subsection*{E New Contributors}\label{new-contributors}}
\subsubsection*{E.1}

\begin{itemize}
\tightlist
\item
  @johannes67890 made their first contribution in\\
  https://github.com/johannes67890/MinionMap/pull/1
\item
  @Spurberino made their first contribution in\\
  https://github.com/johannes67890/MinionMap/pull/28
\item
  @AndreasLN made their first contribution in\\
  https://github.com/johannes67890/MinionMap/pull/31
\item
  @MessiGames30 made their first contribution in\\
  https://github.com/johannes67890/MinionMap/pull/30
\item
  @Hopsasasa made their first contribution in\\
  https://github.com/johannes67890/MinionMap/pull/37
\end{itemize}

\textbf{Full Changelog}:
https://github.com/johannes67890/MinionMap/commits/1.0

